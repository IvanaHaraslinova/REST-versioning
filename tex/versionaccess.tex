\chapter{Version access}
\label{chap:versionaccess}

When the service is versioned using new environment for every new version the version access is straightforward. The version has always the same endpoint address. One environment holds only one version and to choose between versions there is needed to change the environment against which consumer's requests are called. While every environment has its own address.

When the service interface is versioned consumers need to know how to access requred version. Producer of service creates an access point for the new version and consumer has to modify his implementation according to agreed pattern to switch to the new version. 

\section{Access approaches}
Deployment of the versions can be performed various ways. There are two main approaches \cite{applied-soa}:
\begin{enumerate}
  \item Version parameter
  \item Multiple endpoint addresses
\end{enumerate}

\subsection{Version parameter}
Every consumer of the services uses the same endpoint address to access the services. Single service has the same address for any of its implementations. Then then the object of versioning is the implementation. Behind one service which is accessed by one URI can be multiple implementations.
To access a requested version of the implementation a version parameter (also called covenant) can be used. It is present in request message sent by consumer. Thanks to the parameter the request is routed to appropriate implementation version.


The version parameter is carried in request header field. It can be used the existing header or can be implemented new one. From existing headers the \emph{Accept} header is used.


This approach is an advantage for consumers. Release of a new version has minimal impact on them. But on the other side it requires a special attention on version strategy. It can lead to collision of names, database names, etc.


\subsection{Multiple endpoint addresses}
In this approach each service version is deployed independently and has its own endpoint address. This time consumer is obliged to know which service and implementation version he wants to access. When a customer wants to upgrade to a newer version he needs to make changes in the request message. 

The advantage of this approach is elimination of the name collisions and independent deployment of service versions. But the routing is more complex and requires addressing registry for resolving the endpoint addresses.

The schema provide better scalability because the service version is called directly. There is no need to pass through the mediator which solves the selection of a version. Moreover, it reduces coupling between the versions.


\section{Application of access approaches}
There are lots of different theories and opinions about the right approach to implement the versioning. The most discussed are the URI versioning and versioning using the headers. URI versioning represents the Multiple endpoint addresses approach described above and Version in headers represents the Version parameter approach.


\subsection{Multiple endpoint addresses}
Version in URI is broadly used solution how to version. The version is displayed in the URI address:

\begin{lstlisting}
    {api/v2/accounts/}
\end{lstlisting}

The address example suggests that the API is in version \emph{v2}. The request is routed to Accounts service in API v2.

\bigskip

In this approach every version has its own endpoint address. It is hardcoded in version implementation. This approach is easy to understand and comparing to Version in header, it is easier to implement. For different versioning units the URL can contain the version on different place. As was already mention the method versioning requires a change of addressing schema. It's mandatory to specify the method and its version. 

\begin{table}
\caption{Exapmples of version units URL addresses}
\begin{center}
\begin{tabular}{l l l}
Version unit & URL Path  \\ \hline
API & api/v3/customers  \\
Service & api/customers/v3  \\
Method & api/customers/GetCustomer/v3  \\

\end{tabular}
\end{center}
\end{table}

\subsection{Versioning based on headers}
There are several different options of using a header to identify the version of services. Headers are part of the HTTP requests. An existing header or a custom one can be used. From existing headers, the \emph{Accept} header is suitble. It indicates the media types of a request. Creating a custom header offers the creation of header with arbitrary name which lists the version. Version parameter is added to the header to ensure the access to specific version of API.

\subsubsection{Accept header}
The version can be identified by Accept header. "The Accept request-header field can be used to specify certain media types which are acceptable for the response." \cite{website:w3} Accept haeder contains \gls{mime-types} which are typically \gls{xml} or \gls{json}. It can specify the limited subset of accepted media types. Than the request can look like:

\begin{lstlisting}
    {GET /accounts/123 HTTP/1.1}
    {Accept: application/vnd.v2+json}
\end{lstlisting}

The sent request is a GET operation for the account service. The response with representation of the account looks like:

\begin{lstlisting}
    {HTTP/1.1 200 OK}
    {Content: application/vnd.v2+json}
    {
      "id": "123",
      "name": "My Account",
      "type": "Individual",
      "date-created": "28-4-2015"
    }       
\end{lstlisting}

Custom vendor MIME type \emph{vnd.v2} was created. Common feature of Version parameter in header is unchaging URL address. The disadvantage is that customers don't know what type of the response message can expect from the custom MIME type. Media type versioning doesn't provide enough semantic information.


\subsubsection{Custom header}
Provider of services can implement a custom header to define version of the service. It has to be unique, so that it is not already reserved in predefined HTTP headers. An example of custom header is \emph{api-version}. The custom header than has to be implemented in services and has to route the request to appropriate version.

\begin{lstlisting}
    {GET /accounts/123 HTTP/1.1}
    {api-version: 2}
\end{lstlisting}


\subsection{Summary}
On many programming blogs can be found various opinions about approaches to access the version. Completely opposite views on the suitable way to access the resource can be found. Both of the approaches, URL versioning and version parameter in header, have many arguments against, at some poins they violate certain rules (REST constraints, description of resource,..) but they still work in favour of building the API.

\subsubsection{Subjective opinions}
\emph{"If we concur that the URL represents the resource then unless we’re trying to represent different versions of the resource itself then no, I don’t think the URL should change."\cite{website:wrong-ways}}
\bigskip

\emph{"First, introducing version identifiers in the URI leads to a very large URI footprint. This is due to the fact that any breaking change in any of the published APIs will introduce a whole new tree of representations for the entire API.\cite{website:versioning-rest-api}}
\bigskip

\emph{"...the version needs to be in the URL to ensure browser explorability of the resources across versions..."\cite{website:best-practices-rest}}
\bigskip

\emph{"...media types are cheap so we can – and should – have as many as we need. Used properly, content negotiation can be used to solve the problems related to versioning a REST/HTTP web service interface."\cite{website:versioning-rest-web-services}}

\bigskip

It can be said that there are two main groups of programmers. One found their opinion in HATEOAS principle. HATEOAS principle says that the URL should be permalink. URL for the same resource should't change in time. When a new version of API is accessible the resources wasn't changed, the operation work over the same resources consequently the URL address to them shouldn't be changed. This is the main reason why headers sholud be used for versioning. 

On the other side stands a group of speaking are arguments that headers are not the sematnic enough to describe the resource, accept header is hard to test or due to unchanging URL the response for a request can be cached, after sending another request with the same URL but different version in header the wrong version response will be returned from the cache.

\bigskip

These two approaches can be combined, API can be versioned by URL and header at the same time. This solution is using the Stripe API. When they have a backward compatible change they release a new date version which is accessible by custom header. The breaking changes are accessible by versioned URL. The request then looks like: 

\begin{lstlisting}
    {https://api.stripe.com/v1/charges}
    {Stripe-Version: 2015-04-07}
\end{lstlisting}

The version of API is defined in URL and at the same time the request contains a custom header \emph{Stripe-Version} containing the date version.

To conclude there is no the right way to access the version, the choice is up to the preference of one of them. The approach has to be part of the service versioning strategy.

\section{Versioning approach of popular REST APIs}
The versioning is inevitable because the application is changing in time. The companies have to version their REST APIs and each of them can choose one or more principles to implement the access to their resources. 
Here are exapmle of the versioning approaches which are used by well known web applications with REST API. \cite{website:restapi-example}

\begin{table}
\caption{Versioning approach of popular REST APIs}
\begin{center}
\begin{tabular}{|l|l|l|}
\hline
\textbf{API NAME} & \textbf{VERSIONING} &	\textbf{EXAMPLE} \\ \hline

Twillo & date in URI & \\ \hline
Twitter &	URI & \\ \hline
Atlassian & URI & \\ \hline
Google Search	& URI	& \\ \hline
Github API & \begin{tabular}[c]{@{}l@{}}URI/Media Type \\ in v3\end{tabular} & \begin{tabular}[c]{@{}l@{}}Intention is to remove versioning \\ in favour of hypermedia –  \\ current \begin{lstlisting}
application/vnd.github.v3
\end{lstlisting}
\end{tabular} \\ \hline
Azure	& Custom Header	& \begin{lstlisting}
x-ms-version: 2011-08-18
\end{lstlisting} \\ \hline
Facebook & \begin{tabular}[c]{@{}l@{}}URI/optional \\ versioning\end{tabular}	& \begin{lstlisting}
graph.facebook.com/v1.0/me
\end{lstlisting} \\ \hline
Bing Maps &	URI &	\\ \hline
Netflix	& URI parameter	& \begin{lstlisting}
http://api.netflix.com/catalog/
titles/series/70023522?v=1.5
\end{lstlisting} \\ \hline
LinkedIn &	URI &	\begin{lstlisting}
http://api.linkedin.com/v1/
people/~/connections
\end{lstlisting} \\ \hline
Foursquare	& URI	& \begin{lstlisting}
https://api.foursquare.com/v2/
venues/40a55d80f964a52020f31ee3?
oauth_token=XXX&v=YYYYMMDD
\end{lstlisting} \\ \hline
\begin{tabular}{@{}l@{}}Google data API\\ (youtube/spreadsheets/\\ others)\end{tabular}	& \begin{tabular}[c]{@{}l@{}}URI parameter \\ or custom header\end{tabular} & \begin{lstlisting}
"GData-Version: X.0" or "v=X.0"
\end{lstlisting} \\ \hline
Salesforce & \begin{tabular}[c]{@{}l@{}}URI with version \\ introspection\end{tabular} & \begin{lstlisting}
{"label":"Winter '10"
"version":"20.0",
"url":"/services/data/v20.0",} 
\end{lstlisting} \\ \hline
PayPal &	parameter	& \begin{lstlisting}
&VERSION=XX.0
\end{lstlisting}  \\ \hline
Drop Box	& URI & \begin{lstlisting}
https://api.dropbox.com/1/oauth/
request_token
\end{lstlisting} \\ \hline
\begin{tabular}[c]{@{}l@{}}Youtube data \\ API versioning\end{tabular} &	URI	& \begin{lstlisting}
https://www.googleapis.com/
youtube/v3
\end{lstlisting} \\ \hline
\end{tabular}
\end{center}
\end{table}

From the expamle is obvious that none of the approaches is the right one, provders are using all mentioned with the prevalece of URL versioning.

\newacronym{rest}{REST}{REpresentationl State Tranfer}

\newacronym{uri}{URI}{Uniform resource identifier}

\newacronym{hateoas}{HATEOAS}{Hypermedia as the Engine of Application State}

\newacronym{uat}{UAT}{User Accteptance Testing}


\newglossaryentry{service-oriented-design}{
  name=Service-oriented design, 
  description={lalalaa}
}

\newglossaryentry{agile}{
name=agile,
description={"Agile" is related to management of software development. It is a methodology which is practice-based having values, principles and best practices from software modeling. It's based on communication, test-driven design, iteration modeling, documentation, envisioning of architecture and requirements, etc.}
}

\newglossaryentry{monolithic-systems}{
name=monolithic systems,
description={Monolithic systems have single-tiered architecture. They are composed of tight coupled components which are interdependent. There is no possibility to separate any of the component and reuse it.}
}

\newglossaryentry{sla}{
  name=Service-Layer Agreement, 
  description={lalala}
}

\newglossaryentry{xml}{
  name=XML, 
  description={From eXtensible Markup Language, it is a media type lalala}
}

\newglossaryentry{resource-based-model} {
  name=resource-based model, 
  description={/TODO}
}

\newglossaryentry{json}{
  name=JSON,
  description={JSON or JavaScript Object Notation, is an open standard format that uses human-readable text to transmit data objects consisting of attribute–value pairs. It is used primarily to transmit data between a server and web application, as an alternative to XML}
}

\newglossaryentry{http}{
  name=HTTP, 
  description={HyperText Transfer Protocol is a protocol to transfer the HTML (HyperText Markup Language) documents between client and server}
}

\newglossaryentry{adapter}{ 
  name=adapter, 
  description={SOA design pattern. It serves to adapt incompatible interfaces}
}

\newglossaryentry{agnostic-services} {
  name=agnostic services,
  description={Services which are not conscious of the context in which are called, they don't know how the service is implemented or what technology was used}
}

\newglossaryentry{deployment}{
  name=deployment,
  description={}
} 

\newglossaryentry{framework}{
  name=framework,
  description={A tool to ease the software development, in typically con tains supporting libraries, tools and/or best practices for development}
}

\newglossaryentry{session} {
  name=session,
  description={Indicates persistent connection between client and server}
}

\newglossaryentry{webapi} {
  name=ASP.NET Web API,
  description={A framework to biuld HTTP services and a platform fo RESTful application on .NET Framework}
}



\newglossaryentry{hypermedia} {
name=hypermedia,
description={Style of systems which access information via network by hyperlinks, for example the World Wide Web}
}

\newglossaryentry{CRUD}{
name=CRUD,
description={Operations of Create, Read, Update and Delete}
}

\newglossaryentry{deprecation}{
name=deprecation,
description={Deprecated means that the service or service method is still present in new version and can be still used by consumers to do not harm them. Consumers should not count with it but should implement new functionality in short future. Deprecated funcionality will be removed at the point it is no longer used.}
}

\newglossaryentry{scm}{
name=Software Configuration Management,
description={Software Configuration Management is tool used to track the changes in the developed software. It manages changes done by the multiple people.}
}
\newglossaryentry{dll}{
name=DLL,
description={Dynamic Link Library (DLL) is a library of executable functions or data that can be used by a Windows application. Typically, a DLL provides one or more particular functions and a program accesses the functions by creating either a static or dynamic link to the DLL. A static link remains constant during program execution while a dynamic link is created by the program as needed. DLLs can also contain just data. DLL files usually end with the extension .dll,.exe., drv, or .fon. A DLL can be used by several applications at the same time. Some DLLs are provided with the Windows operating system and available for any Windows application. Other DLLs are written for a particular application and are loaded with the application.\ref{website:webopedia}}
}

\newglossaryentry{url}{
name=URL,
description={Uniform Resource Locator (URI) specifies the location of a requested resource on a server}
}

\newglossaryentry{api}{
name=API,
description={Application Program Interface is a set of routines, protocols, and tools for building software applications.\ref{website:webopedia}}
}

\newglossaryentry{mime-types}{
name=MIME types,
description={Multi-purpose Internet Mail Extensions (MIME) defines standard for file types on the Internet, web servers and browser have thair lit of MIME types to transfer files of the same type it spicified way. MIME types have two parts, type and subtype. The type can be \emph{application}, \emph{video}, \emph{image}, the subtype for application can be \emph{application/xml}}
}

\newglossaryentry{web-applications}{
name = web applications,
description = {Web applicatios are appliccations running on the network.}
}

\newglossaryentry{method}{
name=method,
description={Relating the clients server comunication over the network a method is an operation. It is part of the request sent from client to server and defines what operation soul be executed over the data on the server.}
}

\newglossaryentry{documentation}{
name=services documentation,
description={It is a description of services. What are their functions and how to call them. There are described methods available for every service with the templates of URL, requests and responses.}
}

\newglossaryentry{design-patterns}{
name= design patterns,
description = {Design Pattern is a solution for commonly occuring problem in software engineering. Design patterns are reusable and helps avoiding to repeat solving of the same problem.}
}

\newglossaryentry{dal}{
name=DAL,
description={Data access layer usually stand between application layer and Data layer. It mediates communication between them. The DAL prevents the SQL injection (unauthorized access to database) and ansure the following of best practices related to data manipulation.}
}


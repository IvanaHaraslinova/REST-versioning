\newacronym{rest}{REST}{REpresentational State Transfer}

\newacronym{uri}{URI}{Uniform Resource Identifier}

\newacronym{hateoas}{HATEOAS}{Hypermedia as the Engine of Application State}

\newacronym{uat}{UAT}{User Acceptance Testing}

\newglossaryentry{agile}{
name=agile,
description={"Agile" is related to management of software development. It is a methodology which has practice-based values, principles and best practices from software modeling. It's based on communication, test-driven design, iteration modeling, documentation, envisioning of architecture, requirements, etc}
}

\newglossaryentry{monolithic-systems}{
name=monolithic systems,
description={Monolithic systems have single-tiered architecture. They are composed of tight coupled components which are interdependent. There is no possibility to separate any of the components and reuse it}
}

\newglossaryentry{xml}{
  name=XML, 
  description={From eXtensible Markup Language, it is a language to encode documents. It is composed by elements which start and end with tags. An example of such an element is \emph{<name>Joseph</name>}}
}

\newglossaryentry{resource-based-model} {
  name=resource-based model, 
  description={Resource-based model is a design of an architecture which operates over resources. Resource is a typed object containing data}
}

\newglossaryentry{json}{
  name=JSON,
  description={JSON or JavaScript Object Notation, is an open standard format that uses human-readable text to transmit data objects consisting of attribute–value pairs. It is used primarily to transmit data between a server and web application, as an alternative to XML}
}

\newglossaryentry{http}{
  name=HTTP, 
  description={HyperText Transfer Protocol is a protocol to transfer the HTML (HyperText Markup Language) documents between client and server}
}

\newglossaryentry{adapter}{ 
  name=adapter, 
  description={Adapter is a design pattern which serves to transform data, allowing two incompatible interfaces to communicate}
}

\newglossaryentry{agnostic-services} {
  name=agnostic services,
  description={Services which are not conscious of the context in which they are called, therefore they don't know how the service is implemented or what technology was used}
}

\newglossaryentry{framework}{
  name=framework,
  description={A tool to ease the software development, it typically contains supporting libraries, tools and/or best practices for development}
}

\newglossaryentry{session} {
  name=session,
  description={Indicates persistent connection between client and server}
}

\newglossaryentry{hypermedia} {
name=hypermedia,
description={Style of systems which access information via network by hyperlinks, for example the World Wide Web}
}

\newglossaryentry{CRUD}{
name=CRUD,
description={Operations of Create, Read, Update and Delete}
}

\newglossaryentry{deprecation}{
name=deprecation,
description={Deprecated means that the service or service method is still present in a new version and can be still used by consumers. Consumers should not be relying on it and should implement it's replacement functionality in short future. Deprecated functionality will be removed at the point when it is no longer used}
}

\newglossaryentry{scm}{
name=Software Configuration Management,
description={Software Configuration Management is tool used to track the changes in the developed software. It manages changes done by multiple people}
}

\newglossaryentry{dll}{
name=DLL,
description={Dynamic Link Library (DLL) is a library of executable functions or data that can be used by a Windows application. Typically, a DLL provides one or more particular functions. A program accesses the functions by creating either a static or dynamic link to the DLL. A static link remains constant during program execution while a dynamic link is created by the program as needed. DLLs can also contain just data. DLL files usually end with the extension .dll,.exe., drv, or .fon. A DLL can be used by several applications at the same time. Some DLLs are provided with the Windows operating system and available for any Windows application. Other DLLs are written for a particular application and are loaded with the application \cite{website:webopedia}}
}

\newglossaryentry{url}{
name=URL,
description={Uniform Resource Locator (URL) specifies the location of a requested resource on a server}
}

\newglossaryentry{api}{
name=API,
description={Application Program Interface is a set of routines, protocols, and tools for building software applications\cite{website:webopedia}}
}

\newglossaryentry{mime-types}{
name=MIME types,
description={Multi-purpose Internet Mail Extensions (MIME) define standard for file types on the Internet. Web servers and browsers have their list of MIME types to transfer files of the same type it a specified way. MIME types have two parts: type and subtype. Type can be an \emph{application}, \emph{video}, \emph{image}. Subtype for application can be \emph{application/xml}}
}

\newglossaryentry{web-applications}{
name = web applications,
description = {Web applications are applications running on the network}
}

\newglossaryentry{method}{
name=method,
description={In relation to client-server communication over the network a method is an operation. It is a part of the request sent from client to server and defines what operation should be executed over the data on the server}
}

\newglossaryentry{documentation}{
name=services documentation,
description={It is a description of services. What are their functions and how to call them. There are described methods available for every service with the templates of URL, requests and responses}
}

\newglossaryentry{design-patterns}{
name= design patterns,
description = {Design Pattern is a solution for commonly occurring problem in software engineering. Design patterns are reusable and help avoiding repetitive solving of the same problem}
}

\newglossaryentry{dal}{
name=DAL,
description={Data access layer usually stand between application layer and Data layer, mediating the communication between them. DAL prevents the SQL injection (unauthorized access to database) and ensures best practices related to data manipulation}
}

\newglossaryentry{queries}{
name= query strings,
description={Query string is a dynamic part of the URL. It is a search parameter which stands at the end of URL address. It is typically accompanied by \emph{?} character. An example of query string is \emph{www.myweb.com/?ie=UTF-8}}
}

\newglossaryentry{domain-name}{
name=domain name,
description={Domain name is the required part of the URL address. For example \emph{www.mydomain.com}. It is possible to replace it by IP address which is a numerical representation of the address}
}

\newglossaryentry{scripts}{
name=scripts,
description={A script is a list of commands that can be executed without user interaction. A script language is a simple programming language with which you can write scripts. \cite{website:webopedia}}
}



\newacronym{ie}{IE}{Internet Explorer}

\newacronym{esf}{ESF}{Ekonomicko Správní Fakulta}

\newacronym{eu}{EU}{Evropská Unie}

\newacronym{cors}{CORS}{Cross-origin Resource Sharing}

\newacronym{is}{IS}{Informační systém}

\newacronym{muni}{MUNI}{Masarykova Univerzita}

\newacronym{rdp}{RDP}{Remote Desktop Protocol}

\newglossaryentry{ajax}{
  name=AJAX, 
  description={Asynchronní JavaScript a XML označuje způsob vývoje aplikací pomocí asynchronní komunikace mezi prohlížečem a serverem}
}

\newglossaryentry{aspi}{
  name=ASPI, 
  description={,,Automatizovaný Systém Právních Informací'' označuje informační systém vyvíjený společností Wolters Kluwer, poskytující komplexní informace z právníckých oborů}
}

\newglossaryentry{api}{ 
  name=API, 
  description={Aplikační rozhraní (API z anglického \emph{Application Programming Interface}) označuje rozhraní poskytované k integraci programů třetích stran}
}

\newglossaryentry{bundler} {
  name=Bundler, 
  description={Nástroj pro jednoduchou správu a instalaci gemů (balíčků programů programovacího jazyka Ruby)}
}

\newglossaryentry{ci}{
  name=CI,
  description={Průběžná integrace (CI z anglického \emph{Continuous Integration}) označuje souhrn nástrojů použitých k průběžné kontrole zdrojového kódu. Typicky sem patří spouštění testů, kontrola kvality kódu, statická analýza kódu a podobně}
}

\newglossaryentry{cms}{ 
  name=CMS, 
  description={Systém pro správu obsahu (CMS z anglického \emph{Content Management System}) označuje typicky internetovou aplikaci umožňující uživatelům úpravu obsahu. Bývá také označován jako redakční systém}
}

\newglossaryentry{css}{ 
  name=CSS, 
  description={Kaskádové styly (CSS z anglického \emph{Cascading Style Sheets}) je jazyk určený k popisu vzhledu webových stránek}
}

\newglossaryentry{cvs}{
  name=CVS,
  description={Systém ke správě verzí projektu (CVS z anglického \emph{Concurrent Version System}) slouží k ukládání historie verzí zdrojového kódu}
}

\newglossaryentry{deployment}{
  name=nasazení,
  description={Proces instalace projektu na typicky vzdálený server a spuštění případných migračních skriptů a pododbně}
}

\newglossaryentry{framework}{
  name=framework,
  description={Označení pro nástroj ulehčující vývoj software, typicky obsahující podpůrné knihovny, nástroje či popisující správný postup vývoje}
}

\newglossaryentry{mediaqueries}{
  name=@Media-Queries,
  description={Pravidla jazyka CSS umožňující podmínit použití vnořených pravidel dle určíté podmínky (typicky rozlišení monitoru a podobně)}
}

\newglossaryentry{opensource}{
  name=open-source,
  description={Software jehož zdrojový kód je volně dostupný a dle licence i upravitelný}
}

\newglossaryentry{responsive}{
  name=responsivní web design, 
  description={Způsob stylování webových dokumentů, při kterém je brán ohled na různá rozlišení klientských zařízení (telefon, tablet, počítač)}
}

\newglossaryentry{servlet}{
  name=Servlet,
  description={Program v jazyce JAVA, který na straně serveru zpracovává HTTP požadavky}
}

\newglossaryentry{ssh}{
  name=SSH,
  description={Zabezpečený komunikační protokol (z anglického \emph{Secure Shell} používaný v TCP/IP sítích}
}

\newglossaryentry{wysiwyg}{
  name=WYSIWYG, 
  description={Zkratka anglického \emph{,,What you see is what you get''}, doslowně přeloženo jako ,,dostaneš to co vidíš''. Používá se pro označení editorů html kódu, které poskytují formátování pomocí tlačítek a výstup automaticky konvertují do html kódu}
}

\newglossaryentry{url}{
  name=URL,
  description={Jednotný lokátor zdrojů (URL z anglického \emph{Uniform Resource Locator} označuje řetězec znaků definující jedinečné umístění}
}

\newglossaryentry{ruby} {
  name=Ruby, 
  description={}
}

\newglossaryentry{session} {
  name=relace,
  description={Také jako sezení, označuje přetrvávající spojení mezi serverem a klientem}
}

\newglossaryentry{uco} {
  name=UČO, 
  description={Unikátní číslo studenta či zaměstnance vysoké školy}
}

\newglossaryentry{xss} {
  name=XSS, 
  description={Využití bezpečnostních chyb stránky (XSS z anglického \emph{Cross-Site Scripting} za pomoci narušení skriptů stránek a podstrčení změněného kódu či dat}
}

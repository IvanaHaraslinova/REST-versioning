\chapter{Introduction}
\label{chap:introduction}

Service-oriented architecture is a popular architectural approach to application development. The requirements on the market are dynamically changing. Service-oriented applications have properties which respond to the needs of market in better way. They are flexible and incorporate many modern technologies to their development. Service-oriented applications consist of layers, one of which contains services. They provide functionality to consumer. The consumer is in a layer above services and uses them similarly as any customer who is asking for a service in a real world.

Companies oriented this way offer their application as “product as a service”. This product can be consumed by more than one customer. Fulfilling the needs of each of the clients implies modifications of application to suit all of requirements. Customers can ask for changes on the application and service provider needs to make corresponding amendments. Large-scale changes can have impact on all other customers so they can’t be reflected immediately. This is where the challenge of versioning origins from. 

Versioning strategies and approaches are a wide topic. There are many of them, many ideas and applications which are already in use by companies. Some services are versioned by a good strategy and some of them are worse. But it can be said that there is no universal solution to how versioning should be done. 

This thesis is concerned with the services as one of the layers of service-oriented application, focusing on REST-based services. The goal is to analyze the possibilities of REST services versioning and show it on examples. Firstly, it explains what the services are and introduces the REST principles. Then it goes through the versioning strategies and approaches. Finally, this thesis analyzes services versioning techniques in an existing company, summarizing it and proposing eventual improvements.

\section{Summary}

Second chapter contains a short introduction to service-oriented computing. It is a hierarchically highest concept involving service-oriented architecture, design and other principles. The service-oriented architecture (SOA) is described in more detail. There are it’s specifications and advantages. Further in the chapter there is a section about services as a main element of SOA. There are descriptions of what they are and how are they created.

Third chapter deals with the Representational State Transfer (REST). An architectural style according to which services can be designed. REST has its constraints and services have to be designed accordingly to be RESTful. A description of how to meet REST requirements follows. REST is composed by a set of elements which are enumerated and defined. Finally there is an example of REST service.

Next chapter talks about the versioning of services - what causes a need to create a new version and how are the services versioned. There is a description of versioning strategy, talking about the strategical decisions being done before the services are developed. Strategy is then applied during the versioning. 

There are many approaches to service versioning. Chapter five presents two  of different approaches, analyzes them in detail and summarizes their advantages and disadvatages.

Next chapter is about versioning access. Provider can make the services available by various approaches. The concrete techniques are listed with examples applied to REST services. There is a summary of advantages and disadvantages based on the opinion of multiple experienced programmers. Final part of the chapter contains a summarizing table of versioning accesses of well-known REST services.

Seventh chapter analyzes and applies the versioning on a real company. It introduces a company’s architecture and talks about it’s versioning strategy. It also outlines a possible evolution and changes within their services.

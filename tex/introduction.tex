\chapter{Introduction}
\label{chap:introduction}

Service-oriented architecture is popular architectural approach to application development. The requirements on the market are dynamically changing. Applications with service-orientation has properties which better response to the needs of market. They are flexible and have a big amount of modern technologies to be applied to their development. Service-oriented applications consist of layers where one of them contains services. They provide functionality to consumer. The consumer is in layer above and is using the services similarly as any customer which is asking for a service in real world.

Companies oriented this way offer their application as a product. Product can be consumed by more than one customer. Corresponding to the needs of each of the clients implies modifications of application to suit all of requirements. Customers can ask for changes on used application and service provider have to correspond to their requirements. Large-scale changes can have impact on all other customers so the can't be reflected immediately on application. Here origins the challenge of versioning. Versioning strategies and approaches are wide topic. There are many of them, many ideas and applications which are already in use by companies. Some services are versioned having good strategy and some of them are less good. But it can be sait that there is no universal solution how versioning should be done. 

\section{Summary}

This thesis is concerned with the services as one of the layers of service-oriented application. Specially it focuses on versioning of the services and analysis of versioning implementation in corporate environment. 

Second chapter contains a short introduction to service-oriented computing. It is a hierarchically highest concept involving service-oriented architecture, design and other principles. The service-oriented architecture (SOA) is described in more detail. There are its specifications and advantages. Further in chapter there is a section about services as a main element of SOA. There are descriptions of what they are and how are created.

Third chapter deals with the Representational State Transfer (REST). An architectural styles according which services can be designed. REST has its constraints and services has to be designed properly to be RESTful. In the chapter there is a description how to meet REST requirements. REST is composed by a set of elements are enumerated and defined. And finally there is an example of REST service.

Next chapter talks about the versioning of services - what causes a need to create a new version and how the services are versioned. There is a description of versioning strategy. The strategical decisions are done before the services are developed. During the versioning the strategy is applied. There are many approaches to version, two of them are analyzed in detail.

Chapter five is about the versioning access. Provider can make the services available by various approaches. The concrete techniques are listed with their examples and applications on REST services. There is a summary of advantages and disadvantages based on the opinion of experienced programmers. At the end of chapter is a table with example of version access of well-known REST services.

Sixth chapter analyzes and apply the versioning on a real company. It introduces a company's architecture and talk about the versioning strategy. It design a possible evolution and changes within their services.

%What are the services will be described in first part. There is ,. The service oriented architecture (SOA) is described in more detail. %Services are components of the SOA. 


\chapter{Service-oriented computing}
\label{chap:service-oriented computing}
\emph{Service-oriented computing} represents a distributed computing platform. [1] It has its own paradigm, logic, architecture and patterns. It is built on the distributed computing platforms and extends it by new considerations about governance, design layers and technologies suitable for its implementation.
\emph{Service orientation} is a design paradigm, it divides the system into logic units which are separately shaped and can be utilized according to strategic goals and benefits of a requested result.

\section{Service-oriented architecture}
\emph{Service-oriented architecture (SOA)} is a set of best practices for the organization which leads to agile architectural model of the system to meet business needs. The result of its application is an architecture which corresponds to the dynamic market changes. SOA principles do not define any technology and protocols. There are many technologies which could be utilized to create SOA or the proprietary environment could be developed as well. 

The SOA best practices are describing the human behaviour, there is no list of constraints which have to be followed to obtain a service-oriented architecture. The best practices are designed to resolve specific situations which the organization can meet and depending on them could be selected just a subset of appropriate practices which are necessary to apply.

One of the main advantages of a service-oriented architectural style is its ability to efficiently deal with changes. SOA is based on a decomposition of enterprise IT assets and separation of "stable" IT artifacts (services) from "changeable" artifacts (business processes), orchestrating services into IT solutions (processes). (http://msdn.microsoft.com/en-us/library/bb491124.aspx). The speed of changes in the bussines department is to fast to be applied to development and maintance of the monolitic system, where the funcionality is designed on mesure of customer and can't deal with changes often even without the change of design of whole architecture. SOA offers better support when the business requrements change, the changes are refleted into change of the existing business process, of if it's needed the new process is created using the existing services.


\section{Services}
%%\label{sec:services}
Services are the logic units form which is composed the service-oriented design. Every service is standalone object or component. Every service has its own functional context and related capabilities. Each service is deployerd independetly on another one and on the system which use it. It allows the parallel development, one service can be the part of many products of the corporation.
The essence of Service in the SOA context is the business abstraction - that is a representation of funcionality and/or data presented in business context.[Agile architecture].

There are three levels of services in SOA context:
\begin{enumerate}
  \item \textbf{Service implementation} \hfill \\
Service implementation is the code performing the logic.
  \item \textbf{Service interface} \hfill \\ 
The service contract?? for the interface besides on this level, it provide the underlying logic from implementation but do not say anything about it. The service contracts can be web services or RESTful services.
  \item \textbf{Abstracted service = business service} \hfill \\
Abstracted service rapresents the business capability or data. This services can be composed to create a business process, which is the core abstraction that underlies SOA [ZapThing].
\end{enumerate}



%%\subsubsection*{\textbf{Drupal} \hfill \emph{http://drupal.org}} 
%%\label{subsec:drupal}


\section{Techologies}

\subsection{WSDL + SOAP}
\subsubsection{WSDL}
\subsubsection{RPC}
\subsubsection{SOAP}

\subsection{REST}

Representational State Transfer, REST is an architectural style for bulding ditributed hypermedia applications.
Thanks to REST disappeared many issues related to web services, but it comes with some new.

%%\subsubsection*{\textbf{Guacamole} \hfill \emph{http://guac-dev.org/}}
%%\label{subsec:guacamole}
%% \ref{}.
%%    \item \textbf{[název předmětu 1]} \hfill \\
%%    odkaz na stránku předmětu, obsahující pouze název a informace o spolufinancování \gls{eu}
    
%%\begin{table}
%%  \caption{Základní typy entit v Drupalu}
  %%\label{tab:typy-entit}
  %%\begin{tabular}{ | p{3cm} | l | c | c | }
   %% \hline 
    %%Typ entity & Strojový název & Dostupnost polí & Rozšiřitelnost \\ \hline 
    %%Komentář & comment & \checkmark & \checkmark \\ \hline 
    %%Soubor & file &  & \\ \hline 
    %%Slovník & vocabulary &  & \\ \hline 
    %%Uzel & node & \checkmark & \checkmark \\ \hline 
    %%Uživatel & user & \checkmark & \checkmark \\ \hline 
    %%Záznam slovníku & term & \checkmark & \checkmark \\ \hline             
  %%\end{tabular}
%%\end{table}
%% \emph \emph \texttt.


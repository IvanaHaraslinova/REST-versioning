\chapter{Conclusion}
\label{chap:conclusion}
To understand the service versioning it is necessary to know the service-oriented architecture (SOA). It was introduced in at the beginning of the thesis. SOA works with services which were described with their relationship to real-world processes. When a company is going to produce an application or system which is service-oriented, it has to analyze the real-world processes and group them by logical units. Those units are services, the thesis further works with computer network operated web services. 

The thesis outlines Representational State Transfer (REST) architectural style. REST is one of the possible approaches to implementing web services. There are several concepts related to this architecture which need to be followed to obtain RESTful services. 

Versioning is an inevitable part of the service lifecycle. As the real-world changes, the requirements on applications are changing over time as well. The services has to respond to those changes. Every reaction on the change request has to be rooted in versioning strategy. The strategy should be determined during the analytical part. 

Number of approaches to version the services are quite infinite. What is important is a result - provide the service to client. The breaking change require an effort from both to integrate it independently of versioning approach. The main difference between having just one version and more distinct versions accessible at the time has two points of view. From provider side the difference is if he has to wait to client who integrates the service or whether he can deploy new version of API immediately. From client point of view, it can be said, the difference is in amount of accessible versions. 
This thesis analyze two approaches to versioning to demonstrate the differencies and impact of each of them on service consumer and provider. This can be helpful to make an overview of approaches for consideration about the versioning strategy for an API. 

Another part of thesis describe consumers access the services version. The URL addresses should be permanet. This results from the REST constraint Uniform interface and connected HATEOAS aspect. After a change of the interface the new version is created. When the services specific versioning apporach is used, both API can be deployed and run at the same time. These two versions has a particular way how to access them. The common techniques are enumerated. The teqniques can be than further analyzed and customized for particular application. 


This thesis can be further extended several ways. The work presents only analysis of existing versioning techniques. It can be used as a base for creation a versioning strategy of API. Other extension can be deep analysis of Invest s.r.o. API in order to rework the versioning strategy of the company.  


%Every solution has its advantages and disadvatages. Also in service versioning different approaches has the problematic 


